\documentclass[a4paper]{article}

%----------------导入ctex宏包以支持中文
\usepackage{ctexcap}
\usepackage{setspace}
\usepackage{geometry}
\usepackage{amsmath}
\usepackage{paralist}
\usepackage{graphicx}
\usepackage{fancyhdr}

%-----------------页眉页脚设置
\pagestyle{fancy}
\lhead{Lecture 1}
\chead{Statistical Learning Theory}
\rhead{Bloomberg ML EDU}
\lfoot{}
\cfoot{\thepage}
\rfoot{}
\renewcommand{\headrulewidth}{1pt}
\renewcommand{\footrulewidth}{0pt}

%-----------------调整页边距
\geometry{top=1in, bottom=1in, left=1.25in, right=1.25in}

%-----------------设置字体
\setCJKmainfont[BoldFont=华文中宋, ItalicFont=楷体]{宋体}
\setCJKfamilyfont{song}{宋体}
\newcommand{\song}{\CJKfamily{song}}
\setCJKfamilyfont{kai}{楷体}
\newcommand{\kai}{\CJKfamily{kai}}

%-----------------标题设置
\CTEXsetup[name={,},number={\chinese{section}},format={\Large\raggedright\kai}]{section}

%-----------------行距调整
\setstretch{1.5}

\title{Lecture 1 \quad Statistical Learning Theory}
\author{\kai{Alex\quad 2019年7月21日}}
\date{}

\begin{document}
	\maketitle
	
	\section{决策理论:宏观视角}

	统计学习理论(Statistic Learning Theory)可以看做是决策理论(Decision Theory)的子类或衍生。因此在进入统计学习理论的学习前,有必要对决策理论进行一些阐述。那么,什么是决策理论?\\
	
	\noindent\fbox{%
		\parbox{\textwidth}{%
			\begin{center}
				\textbf{Decision theory is about finding "optimal" actions, \\ under various definitions of optimality.}
			\end{center}
		}%
	}\\
	
	即:决策理论的主要内容是,在确定“什么是最优”的前提下,寻找“最优的”actions。那么,什么是actions?
	
	\subsection{什么是action?}

	\noindent\fbox{%
	\parbox{\textwidth}{%
		\begin{center}
			\textbf{An action is the generic term for what is produced by our system.}
		\end{center}
		}%
	}\\
	
	一个action是我们所构建的系统的产出。例如,对于一个预测未来3小时候后台风位置的系统,它的action可能是一个经纬度坐标(当然也可能是一个关于经纬度的概率分布等等)。
	
	对于系统的actions,我们需要一个能够对其进行打分的评价准则(Evaluation Criteria)。只有这样,我们才能评价我们所构建的系统的优劣。例如,对于刚刚提到的台风位置预测系统,假设action是一个经纬度坐标,我们则可以用3小时后台风的真实位置与action的距离作为评价准则——距离越大,则说明action越不好,那么也可以间接反映我们的预测系统其实不太准确。综上,评价准则对于决策理论十分重要。
	
	系统所有可能的actions的集合称为Action Space。在模型化现实问题时,①确定Action Space与②确定评价准则是最先进行的两个步骤。之后,我们则需要明确输入(input)、结果(outcome)和对应的输入空间(input space)、结果空间(outcome space)。

	\subsection{输入与结果}	
		
	输入,即input,指传入系统的数据。在统计学中,输入又被称为covariate。结果,即outcome,或称label,指输入所对应的“正确答案”,是what  actually happen。特别注意,outcome虽然也可以称为output,但它并不是系统的output——系统的output是上文提到的action!
	
	例如,对于银行的反欺诈系统,input可能是一位客户的信息,如:年龄、性别、历史贷款情况、历史还款情况等。outcome是该客户的真实身份判定,假设该客户是事实上的“坏人”,那么outcome即为“坏人”,这是确定的、唯一的。action是系统基于input的输出,即预测。预测可能出错,即action可能为“好人”。并且,不同的系统对于同一个人的预测也可能不同。因此,我们说outcome like “right answer” is what actually happen,但action则是系统的输出,并不等同于客观事实。
	
	\subsection{继续建模}		
	
	综上,我们可以将问题模型化为(假定y与a独立):

	\begin{itemize}
		\setstretch{0.8}
		\item 观察,得到input x;
		\item 系统根据input得到一个action a;
		\item 观察,得到outcome y;
		\item 根据评价准则,结合y,对a进行评价:$l(a,y)$。
	\end{itemize}

	上述问题对应三个空间:
	
	\begin{itemize}
		\setstretch{0.8}
		\item Input space,记做$\mathcal{X}$,表示由所有可能输入所构成的集合;
		\item Action space,记做$\mathcal{A}$,表示由系统所有可能输出所构成的集合;
		\item Outcome space,记做$\mathcal{Y}$,表示由所有可能结果所构成的集合。
	\end{itemize}
	
	例如,对于线性回归问题,input space是$R^d$,action space是$R$,outcome space也是$R$。对于Logistic回归问题,input space是$R^d$,action space是$[0,1]$,outcome space是$\{0,1\}$。
	
	定义了三大空间后,我们接着定义决策函数(decision function)与损失函数(loss function):\\
	
	\noindent\fbox{%
	\parbox{\textwidth}{%
		\begin{center}
			\textbf{A decision function gets input $x \in \mathcal{X}$ and produces an action $a \in \mathcal{A}$,that is:\\
			$f: \mathcal{X} → \mathcal{A}$;\\
			A loss function evaluates an action in the context of the outcome y,that is:\\
			$l: \mathcal{A} × \mathcal{Y} → \mathcal{R}$。\\}
		\end{center}
		}%
	}\\	
	
	决策函数即系统,它根据输入x,确定action a作为预测。损失函数以action a与outcome y作为它的输出,得到一个数字作为对该action a的评价。但损失函数仅仅是对一个action的评价——如何评价一个决策函数?这就是统计学习理论所关注的核心问题。

	\section{统计学习理论}	
	
	\subsection{Risk:泛化误差}
	
	统计学习理论有两个基本假设:第一,action a与outcome y独立;第二,存在联合分布$P_{\mathcal{X} × \mathcal{Y}}$,所有的数据(x,y)都是从该联合分布中i.i.d产生的,即样本满足"独立同分布"的特点。
	
	如果一个决策函数$f(x)$是“好的”,那么说明该决策函数在联合分布上的平均表现还不错:\\
	
	\noindent\fbox{%
	\parbox{\textwidth}{%
		\begin{center}
			\textbf{$f(x)$ does well on average, or $l(f(x), a)$ is usually small.}
		\end{center}
		}%
	}\\	

	在统计学习理论中,我们用决策函数的“risk”表征它的平均表现,等同于泛化误差:\\
	
	\noindent\fbox{%
	\parbox{\textwidth}{%
		\begin{center}
			\textbf{The risk of a decision function $f: \mathcal{X} → \mathcal{A}$ is:\\
			$R(f) = E\Big[l(f(x),y)\Big]$\\
			In words, it's the expected loss of $f$ on a new example (x,y) drawn randomly from distribution $P_{\mathcal{X} × \mathcal{Y}}$.}
		\end{center}
		}%
	}\\	
	
	在所有可能的决策函数中,一定存在一个$f$的risk最小。该risk最小的决策函数被称为贝叶斯决策函数:\\
	
	\noindent\fbox{%
	\parbox{\textwidth}{%
		\begin{center}
			\textbf{A Bayes decision function $f^*: \mathcal{X} → \mathcal{A}$ is a function that achieves the minimal risk among all possible functions:\\
			$f^* = \underset{f}{\arg\min}\ R(f)$\\
			where the minimum is taken over all functions from $\mathcal{X}$ to $\mathcal{A}$.
			}
		\end{center}
		}%
	}\\
	
	贝叶斯决策函数的risk被称为Bayes risk,它代表决策函数能够达到的最好的情况,是一种“标杆”。有时我们也把贝叶斯决策函数称为target function。下面是两种情景下的贝叶斯函数的形式:
	
	\begin{itemize}
		\setstretch{0.8}
		\item 最小二乘回归:评价准则是平方误差,贝叶斯函数为$f^*(x) = E[y|x]$;
		\item 多元分类问题:评价准则是0-1误差,贝叶斯函数为$f^*(x)=\underset{1\le k \le K}{\arg\max\ P(y=k | x)}$。
	\end{itemize}
	
	然而,我们无法真正解出贝叶斯决策函数,因为我们没有办法计算risk——由于我们仅仅知道有这样一个产生资料的联合分布,但并不知道该联合分布的具体信息,因此risk无法计算。所以我们需要“估计”!

	\subsection{Empirical Risk:经验误差}	
	
	我们现有的数据是从未知分布中产生的样本:$D_n = \{(x_1, y_1), \cdots,(x_n,y_N)\}$。定义决策函数$f$的经验风险(empirical risk):\\
	
	\noindent\fbox{%
	\parbox{\textwidth}{%
		\begin{center}
			\textbf{The empirical risk of $f: \mathcal{X} → \mathcal{A}$ with respect to $D_n$ is:\\
			$\hat{R}_n(f) = \frac{1}{n}\sum_{i=1}^n l(f(x_i), y_i)$
			}
		\end{center}
		}%
	}\\
	
	根据大数定理,当n趋于无穷大时,$\hat{R}_n(f)$依概率收敛于$R(f)$。因此,经验风险是risk的一个相合估计量。同时,很容易证明,经验风险也是risk的无偏估计量。因此,用经验风险去估计risk是一个可行的方法。既然经验风险可以计算,那么我们就可以进行经验风险最小化(empirical risk minimization):\\
	
	\noindent\fbox{%
	\parbox{\textwidth}{%
		\begin{center}
			\textbf{A function $\hat{f}$ is an empirical risk minimizer if:\\
			$\hat{f} = \underset{f}{\arg\min}\ \hat{R}_n(f)$\\
			where the minimum is taken over all functions.
			}
		\end{center}
		}%
	}\\
	
	经验风险最小化的坏处是:它使得最后得到的$\hat{f}$好像仅仅“记住”了数据——在训练数据上不犯任何错误,但这样往往说明它的泛化能力(generalize)很差——对于新的、没有见过的数据的预测能力很差。一种修改方法是,加入一些平滑的因素——smooth flavor,使得决策函数的变化不要特别剧烈。具体而言,我们可以使用Constrained ERM:在所有可行的决策函数中,拿出一个子集,即拿出一部分决策函数出来进行选择:\\

	\noindent\fbox{%
	\parbox{\textwidth}{%
		\begin{center}
			\textbf{Instead of minimizing empirical risk over all decision functions.}
		\end{center}
		}%
	}\\
	
	这“一部分决策函数”,构成假说空间(hypothesis space):\\

	\noindent\fbox{%
	\parbox{\textwidth}{%
		\begin{center}
			\textbf{A hypothesis space $\mathcal{F}$ is a set of functions mapping $\mathcal{X} → \mathcal{A}$. It is the collection of decision functions we are considering.}
		\end{center}
		}%
	}\\
	
	在hypothesis space中,我们有risk最小的hypothesis,记作:
	$$
	\hat{f}_n = \underset{f \in \mathcal{F}}{\arg\min}\ \hat{R}_n(f)
	$$
	
	也有经验风险最小的hypothesis,记作:
	$$
	f^*_{\mathcal{F}} = \underset{f \in \mathcal{F}}{\arg\min}\ R(f)
	$$

\end{document}